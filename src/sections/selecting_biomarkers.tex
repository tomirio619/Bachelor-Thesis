% !TeX spellcheck = en_US
% !TeX root = ../MS_analysis_thesis.tex

\section{Selecting Relevant Biomarkers} \label{section:Selecting Relevant Biomarkers}
As we have seen in \Cref{section:relevant biomarkers}, there is a large collection of biomarkers that could be related to good and bad days. In this section, we will identify which biomarkers could be useful. 
This will be done by looking at the use of biomarkers in other diseases for longer periods of time. 
We must also look at current technology being able to measure the biomarker.

\subsection{Limiting the Set of Relevant Biomarkers}
There are couple of biomarkers for which we decided they could not be used in our research.
This applies to the following biomarkers: vitamin D, weather, neurogenesis, stress, `sitting, walking and lying pattern' and HRV.
We do actually see potential for each individual biomarker in future research.
However, currently those biomarkers are not easy to measure, have not been investigated thoroughly, have not been used in practice or cannot be measured by consumer-available devices.
All of these reasons made us choose a specific selection of biomarkers.
Before looking into this selection, we will describe for each of these biomarkers listed above why we decided not use them in our research.
%
\begin{itemize}
	\item For measuring vitamin D, we would need some sort of advanced sensor that could determine the amount of vitamin D in the blood. 
	Although such devices exist, this would force our participants to manage multiple devices simultaneously. 
	This would add unnecessary complexity.
	While writing this thesis, there did not actually exist any consumer-available device that could measure vitamin D (as far as we know).
	All these reasons made us not to choose this biomarker.
	
	\item Neurogenesis is a new research topic that is currently being explored.
	As we have seen in \Cref{subsection: Neurogenesis}, certain kind of activities and foods are influencing this process.
	Because the topic is relatively new, there is no data from which we can see how a certain type of food or activity influences neurogenesis. 
	If this data was present however, this would require more user intervention: users would need to input their meals into the application.
	This seems undesirable.
	This observation of unwanted complexity ruled out neurogenesis as a biomarker.
	
	\item As seen in \Cref{subsection: Stress}, stress and arousal are partly intertwined. 
	We have seen that the measurement of sweat on the skin using a dedicated sensor can be used to measure levels of arousal. 
	Some activity trackers are able to measure this sweat production on the skin, but not all of them. 
	Because the functionality was not present on all the activity trackers we had available for our experiment, we chose not to include this biomarker.
	
	\item Activity classification is something that most activity trackers can do.
	However, the classification of positions (sitting / walking / lying pattern) is not a basic functionality for currently available activity trackers.
	Due to this currently lacking feature of activity trackers, we decided not to include this biomarker as well.
	
	\item Weather data is freely available, and given that most patients prefer certain temperatures this is a promising biomarker. Due to time restrictions, we decided not to include this biomarker in our research.
	
	\item Despite the restricted knowledge on HRV, it appears to be a reliable biomarker for responses to stress. 
	While acute stress only influences HRV for a short term, chronic stress can even affect HRV during sleep. 
	This was found in \cite{hall2004acute}. 
	These stress related changes in HRV caused significant decreases in sleep maintenance, which is the ability to remain at sleep during night.
	Thus, the effects of some forms of stress on HRV can be seen for an extended period of time. 
	Despite being a very useful biomarker, HRV cannot be measured using current available smartwatches.
	It requires advanced technology like ECG to get an accurate approximation of HRV.
	It is very likely that smartwatches in the future will have the ability to measure this indicator accurately.
	However, due to the current restriction of technology available for consumers, we will not select this biomarker.
\end{itemize}
%

\subsection{Selected Biomarkers}
Fatigue seems to be one of the most common symptom of MS.
The results from the \textbf{2MWT} can be used as a reliable and valid measure of the physical function of patients. 
This is only possible when the test is standardized.
A problem with the 2MWT is that there is a lack of reference values.
This limits the interpretation of the results.
In \cite{selman2014reference}, this limitation was addressed by establishing a reference equation to predict the distance walked in the 2MWT.
This was done for healthy participants.
The equation was as follows:
%
\begin{align*}
\text{2MWT}_{\text{predicted}} &= 252.583 - (1.165 \times \text{age}) + (19.987 \times \text{gender}^*) \\
& \text{gender}^* = 
\begin{cases*}
1 & \text{if } \text{male}\\
0 & \text{if } \text{female}
\end{cases*}
\end{align*}
%
The equation was found to be highly reproducible in healthy subjects.
It might be possible to come up with an equation for MS patients that is capable of giving an estimation of the distance walked. 
This equation could have additional parameters, like the type of MS the patient has and the current severity of the disease.

Standardization of the 2MWT in the pilot is something that requires some more attention.
Because the lack of a supervisor during this test, people might have different expectations on how to perform the walking test.
There is documentation available in the `Mijn Kwik' app about the test, but if people put this text in the same perspective as terms of use, it is not likely that this documentation is read thoroughly.
Despite these uncertainties, the results of these tests seem promising as a potential biomarker.
For MS, both the 6MWT and the 2MWT \cite{gijbels2010predicting} were tested on relevance of habitual walking performance (HBW).
Both the 2MWT and 6MWT were the best predictors among all the other tests for habitual walking performance, confirming its high potential.

Your \textbf{resting heart rate} (RHR) (\Cref{section:Resting Heart Rate}) can be influenced by a lot of factors, including stress.
When being exposed to stress for a long period, your body is constantly functioning in a high gear and therefore affecting your heart rate.
Stress can also reduce your energy, sleep and make you feel cranky.
Stress is also playing a big role in MS (\Cref{subsection: Stress}).
Although the use of resting heart rate from heart rate data from wearables in research is a new concept, it can prove useful.
Combined with data from the questionnaire app, we can take a closer look at the relation between heart rate and the indicated quality of the next day.

There have been some follow up studies on the effect of elevated resting heart rate on the risk of all-cause mortality.
In \cite{jensen2013elevated}, this effect was researched on 2798 healthy middle-aged men in a 16-year follow up study.
An inverse relation was found between resting heart rate (RHR) and physical fitness.
An increased RHR was also related to mortality, independent of the physical fitness of a person

During \textbf{sleep} (\Cref{section:Sleep}), you lay the foundation for your next day.
You will immediately notice the impact of lack of sleep from the previous night.
Sleep also plays an important role in your mood.
In \cite{totterdell1994associations}, the relations between sleep and mood were investigated.
It was found that sleep was more related to subsequent well-being than prior well-being.
Also an earlier onset of sleep was related to a better mood the next day. 
Earlier onset of sleep was also a better predictor of mood than sleep duration.
This seems to be in conflict with the general opinion that duration of sleep has the most impact on your mood.

However, in \cite{rehagen2015short} the effect of a short sleep duration on the intake of nutrients and mood was investigated.
Undergraduate college students were used in this experiment.
It was found that the total mood disturbance (TMD) score was linked to sleep duration and sleep quality. 
The Pittsburgh Sleep Quality Index (PSQI) was used to measure the quality and pattern of sleep. 
It also includes questions to estimate the sleep duration.
Participants that had a sleep duration of at least 6 hours during the night had lower TDM scores $(p = 0.0091)$ and therefore experienced a more positive mood state the next day.
This suggests that a short sleep duration and a lower sleep quality have a negative impact on the mood state.

In \cite{wichniak2013sleep}, sleep as a biomarker for depression was investigated.
Changes in a person's sleep pattern also seem to be important for diagnosing a subtype of depression, regardless of being positive (more sleep or an improvement of sleep quality) or negative (decrease in sleep or worsening of sleep quality).
Sleep (and specifically sleep duration) can therefore be used as a biomarker to determine the increased risk for depression, which influences the quality of upcoming days.