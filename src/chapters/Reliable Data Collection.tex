% !TeX spellcheck = en_US
% !TeX root = ../BachelorThesis.tex

\section{Reliable Data Collection} \label{section: Reliable Data Collection}
There is a whole range of different activity trackers on the market that can be used to capture health related data like distance walked or run, burned calories and sometimes even heart rate and quality of sleep. 
These trackers have been developed to increase the insight of an individual into their physical activities throughout the day. 
In order to use this data for research purposes, we must determine the validity and reliability of the data from these activity trackers and other related devices.

In \cite{kooiman2015reliability}, ten activity trackers available for consumers with the ability to measure step count were tested on healthy people.
Testing was done both under laboratory conditions (using a treadmill) and under free-living conditions (on one working day between 9.00 and 16:30) using different groups. 
Besides the Omron, Moves app and the Nike+ Fuelband, most trackers showed a high reliability. 
This was analyzed using test-retest analysis%
\footnote{Test-retest analysis is used to assess the consistency of a measurement from one time to another \cite{trochim2006types}. 
The consistency can be estimated by administering the same test to the same sample on two different moments.} with Intraclass Correlation Coefficient (ICC)%
\footnote{The Intraclass Correlation Coefficient (ICC) is used to quantify the degree of similarity between two or more values repeatedly measured on a continuous scale \cite{koch1982intraclass}.}.
The validity of the trackers was measured by comparing each of the trackers using the most accurate test possible (gold standard) under laboratory and under free-living conditions.

Various studies have been published on the validity and reliability of different activity monitors \cite{bassett2012calibration, chen2012re, butte2012assessing, freedson2012assessment}. These devices are becoming widely accepted in the research field.
In \cite{evenson2015systematic}, the evidence for validity and reliability of two different activity trackers (Fitbit and Jawbone) was systematically reviewed.
This was done for the measurement of sleep, energy expenditure, physical activity, steps and distance.
The study showed that there was a higher validity of steps and physical activity.
There was a lower validity of sleep and energy expenditure. 
Except for the measurement of physical activity, a high inter-device reliability was found.
However, in 7 of the 22 studies, missing or lost data was reported.
Some of the lost data was due to the validation criterion measure and not due to the trackers. 
Other lost data were attributable to research errors.
It was also mentioned that one should expect data loss when using activity trackers, and that software updates of these trackers can influence measurements from these devices.
These influences and the resulting risks should be taken into account during our data gathering and analysis.

In \cite{lee2013validity}, consumer-based physical activity monitors were studied to examine the validation of the measurement of energy expenditure.
This was done under semi-structured free-living conditions. 
Eight different activity monitors were worn during a 69-minute protocol, based on both free living and structured activities. 
Most of the consumer based monitors gave results similar to the Actigraph monitor, which is a research-grade monitor most commonly used in the field. 
This outcome is promising for the use of wearables in our experiment.