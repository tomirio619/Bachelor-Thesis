% !TeX spellcheck = en_US
% !TeX root = ../BachelorThesis.tex
\chapter{Long Term Effects of Wearables} \label{chapter: Long Term Effects}

\section{Privacy}
\lettrine[lhang = 0.4, findent=-60pt, lines=7]{\textbf{
		\initfamily \fontsize{40mm}{40mm} \selectfont A
		\normalfont}}{lthough}
we still have to find out if wearables will help people with MS to get more insight in the quality of upcoming days, there must be looked at the effects from wearing these devices extensively.
One of these aspects is privacy.
Because all of the information is stored on the device and in a database, data loss can have an enormous impact on the user. 
Companies providing these devices have a dominant position.
What if the gathered data is sold for money?
This can really effect patients in taking a health insurance.
Not every patient wants their health insurance company to know in which stage of the disease they are currently residing.
That data loss is a potential threat is shown in a research into the security of smartwatches by HP \cite{hp2015smartwatches}.
In total, ten popular smartwatches were investigated. 
During this research, it was found that some smartwatches were sending information to third parties.
In 7 of the 10 smartwatches, information was send unencrypted to the server.

Another aspect is the influence of wearables on your choices during the day.
Because almost everything is monitored, you cannot deny the patterns that can be made visible using the gathered data.
For example, if you see that your calorie intake is sufficient for this day, you may decide to skip that snack you normally would have eaten.
Of course, this is good for your health, but it puts some restrictions on your freedom, without you knowing.
Next thing you know, you will receive a warning about your calorie intake.
Similar things are already happening on Android devices.
When users of Android devices put their volume too high, they will receive a warning saying that ``listening at high volume for a long period may damage your hearing''.
Although looking at all these small steps in technology individually makes them look harmless, combined they provide a stunning amount of information. 
It seems that people are not aware that this ever increasing amount of data is also continuously increasing the insight in customers and all of their patterns. 
Because people may not be aware of this, this increasing insight in their lives could be unwanted and thus remains something that we have to watch out for.

\section{Health}
The intention of activity trackers is that they are worn 24/7. 
While wearing it, the device is constantly syncing data to your mobile phone.
This makes use of Bluetooth, a wireless technology using the microwave frequency spectrum.
Because people tend to sleep with their hands near their heads, this can cause a significant exposure of the brain during nighttime.
When looking for research on the long term health effects of Bluetooth, it is remarkable that there are hardly any papers available. 
This might be due to the interests of the phone industry in the use of this technology.
Looking into the specification of Bluetooth devices, we see that the maximum output for devices classified in the highest Bluetooth power class is still lower than the lowest powered mobile phones \cite{hietanen2005occupational}.
This implicitly suggests that the exposure to radiation emitted from devices using Bluetooth is save. 
However, some people question this implicit safety.
Because more and more devices are making use of Bluetooth, people will be exposed to higher levels of radiation. 
This makes people question how safe the exposure to this increasing amount of radiation is: should we not reduce the amount of radiation when the levels of radiation we are exposed to are increasing?
Although most smart watches and activity trackers are making use of Bluetooth and WiFi to function, there are also some brands that have a cellular chip in their devices.
Wearing activity trackers with a cellular chip is like wearing a cellphone on your wrist.
The W.H.O has a fact sheet about the relation between public health and mobile phones. 
On \cite{who2014mobilephones} we can read that there have been several longitudinal studies were the long-term risks from radiofrequency exposure were investigated.
Although some studies are still ongoing, the results were not consistent.
Because we cannot rule out the possibility of radiofrequency electromagnetic fields as being carcinogenic, they are are classified as possibly carcinogenic.
This indicates that a causal association between radiofrequency electromagnetic fields and cancer is possible, but that confounding, bias or chance cannot be ruled with any certainty.