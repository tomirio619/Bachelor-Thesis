% !TeX spellcheck = en_US
% !TeX root = ../BachelorThesis.tex
\begin{abstract}
	Multiple sclerosis (MS) is a disease where the day quality of a patient can vary a lot.
	Therefore, it is hard to predict whether upcoming days will be `good' or `bad' days.
	By identifying relevant biomarkers for patients using data from wearables, we might be able to predict this day quality.
	These predictions can be used to get more personal advice on and insight in MS.
	
	After we identified relevant biomarkers using literature research, we chose to research the following three biomarkers: Two-Minute Walk Test (2MWT), resting heart rate (RHR) and sleep duration.
	Using statistical analysis on each of these biomarkers, we investigated the correlation between the data belonging to this biomarker and the rating of the day after indicated by each of the participants.
	Although the period in which we gathered the data was reasonable, the amount of data was lower and of lesser quality than expected.
	This was especially the case for data that only could be gathered when participants entered specific information in the application that we used during our experiment, 
	In the end, we did not find any evidence for a relation between each individual biomarker and the day rating for the next day.
	We did find that the process of gathering data from research subjects should preferably be fully automated (if possible).
	This is because participants tend to forget or ignore instructed tasks, although these tasks are essential for the data that will be used in research.
	Participant adherence was one of the major factors impacting the quality of our data and should be taken into account in future research.
\end{abstract}
